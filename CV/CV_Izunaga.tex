 %-------------------------------------------------
% Title:
% C.V.
% Revision history
% 2018-06-11 by Y. Izunaga
%-------------------------------------------------
\documentclass[dvipdfmx,9pt,a4paper]{article}
%----- packages -------------------------------
\usepackage{enumerate}
\usepackage{fancyhdr}
\usepackage{lastpage}
\usepackage{otf}
\fancypagestyle{mypagestyle}{%
\lhead{}%ヘッダ左を空に
\rhead{}%ヘッダ右を空に
\cfoot{\thepage/\pageref{LastPage}}%フッタ中央に"今のページ数/総ページ数"を設定
\renewcommand{\headrulewidth}{0.0pt}%ヘッダの線を消す
}
%----- size --------------------------------------
\setlength{\topmargin}{0pt}
\setlength{\oddsidemargin}{10pt}
\setlength{\evensidemargin}{10pt}
%\setlength{\textheight}{620pt}
\setlength{\textheight}{650pt}
\setlength{\textwidth}{435pt}
\setlength{\footskip}{28pt}
\renewcommand{\baselinestretch}{1.1}
\setlength{\headsep}{30pt}

%----- page style -----------------------------
\pagestyle{mypagestyle}
%-------------------------------------------------

\title{Curriculum Vitae}
\author{Yoichi Izunaga}
\date{\today}
%------ includeonly ---------------------------
%\includeonly{}
%-------------------------------------------------
\begin{document}

%---title---
\maketitle
\thispagestyle{mypagestyle}


%---Position
\section*{\underline{Position}}
\begin{description}

\item[Apr.~2021--Present] Lecturer, Kyushu University, Faculty of Economics, Department of Economic Engineering

\item[Apr.~2019--Mar.~2021] Assistant Professor, Kanagawa University, Faculty of Engineering, Department of 

\item[Jul.~2017--Mar.~2019] Assistant Professor, University of Tsukuba, Faculty of Business Sciences

\item[Apr.~2016--Jul.~2017] Researcher, The Institute of Behavioral Sciences, Information Systems Research Division

\end{description}

%---Edu
\section*{\underline{Education}}
\begin{description}

\item[Mar.~2016] Doctor of Engineering, University of Tsukuba, Graduate School of Systems and Information Engineering

\item[Mar.~2011] Master of Economics, Yamaguchi University, Graduate School of Economics

\item[Mar.~2009] Bachelor of Economics, Yamaguchi University, Faculty of Economics

\end{description}


%---papers---
\section*{\underline{査読付き学術雑誌論文}}

\begin{enumerate}

\item Satoshi Takahashi, \underline{Yoichi Izunaga}, and Naoki Watanabe (2021):\\
{\bf An experimental study of VCG mechanism for multi-unit auctions: Competing with machine bidders.}\\
{\it Evolutionary and Institutional Economics Review}.

\item \underline{Yoichi Izunaga}, Tomomi Matsui, and Yoshitsugu Yamamoto (2020):\\
{\bf A doubly nonnegative relaxation for modularity density maximization.}\\
{\it Discrete Applied Mathematics}, Vol.275, pp.69--78.

\item Satoshi Takahashi, \underline{Yoichi Izunaga}, and Naoki Watanabe (2019):\\
{\bf VCG mechanism for multi-unit auctions and appearance of information: An experiment.}\\
{\it Evolutionary and Institutional Economics Review}, Vol.16, No.2, pp.357--374.

\item Keisuke Sato and \underline{Yoichi Izunaga} (2019):\\
{\bf An enhanced MILP-based branch-and-price approach to modularity density maximization on graphs.}\\
{\it Computers \& Operations Research}, Vol.106, pp.236--245.

\item Satoshi Takahashi, \underline{Yoichi Izunaga}, and Naoki Watanabe (2018):\\
{\bf An approximation algorithm for multi-unit auctions: Numerical and subject experiments.}\\
{\it Operations Research and Decisions}, Vol.28, No.1, pp.95--115.

\item \underline{Yoichi Izunaga} and Yoshitsugu Yamamoto (2017):\\
{\bf A cutting plane algorithm for modularity maximization problem.}\\
{\it Journal of the Operations Research Society of Japan}, Vol.60, No.1, pp.24--42.

\item \underline{Yoichi Izunaga}, Keisuke Sato, Keiji Tatsumi, and Yoshitsugu Yamamoto (2015):\\
{\bf Row and column generation algorithms for minimum margin maximization of ranking problems.}\\
{\it Journal of the Operations Research Society of Japan}, Vol.58, No.4, pp.394--409.

\item \underline{伊豆永洋一} (2011):\\
{\bf 最小取引単位を考慮したポートフォリオ最適化.}\\
経営情報学会誌, Vol.20, No.3, pp.185--201.

\end{enumerate}

%---proceedings---
\section*{\underline{査読付き会議論文}}

\begin{enumerate}

\item Kotofumi Inaba, \underline{Yoichi Izunaga}, and Yoshitsugu Yamamoto (2016):\\
{\bf A Lagrangian relaxation algorithm for modularity maximization problem.}\\
{\it Operations Research Proceedings 2014}, pp.241--247.

\item \underline{Yoichi Izunaga}, Keisuke Sato, Keiji Tatsumi, and Yoshitsugu Yamamoto (2016):\\
{\bf Row and column generation algorithm for maximization of minimum margin for ranking problems.}\\
{\it Operations Research Proceedings 2014}, pp.249--255.

\end{enumerate}


%---others---
\section*{\underline{その他の論文}}

\begin{enumerate}

\item 猿渡康文,\underline{伊豆永洋一},鵜飼孝盛,蔭山康太 (2019):\\
{\bf 空域編成に対する2つの最適化アプローチ.}\\
{航空無線}, No.101, pp.29--36.

\item 成島康史, 田中未来, Phung-Duc Tuan, \underline{伊豆永洋一}, 鵜飼隆盛, 奥野貴之, 黒沢健, 田中健一 (2019):\\
{\bf 本部SSOR2018開催報告.}\\
{オペレーションズ・リサーチ経営の科学}, Vol.64, No.3, pp.147--155.

\item \underline{Yoichi Izunaga} and Keisuke Sato (2018):\\
{\bf A bounding algorithm for selective graph coloring problem.}\\
{\it RIMS Kokyuroku}, Vol.2069, pp.84-94.

\item \underline{Yoichi Izunaga}, Tomomi Matsui, and Yoshitsugu Yamamoto (2016):\\
{\bf A doubly nonnegative relaxation for modularity density maximization.}\\
{\it RIMS Kokyuroku}, Vol.1981, pp.84-97.

\item \underline{Yoichi Izunaga}, Keisuke Sato, Keiji Tatsumi, and Yoshitsugu Yamamoto (2015):\\
{\bf Row and column generation algorithm for minimum margin maximization of ranking problems.}\\
{\it RIMS Kokyuroku}, Vol.1931, pp.18-30.

\item 稲葉言史,\underline{伊豆永洋一},山本芳嗣 (2015):\\
{\bf モジュラリティ最大化問題に対するラグランジュ緩和法.}\\
{数理解析研究所講究録}, 1931巻, pp.1-17.

\item \underline{伊豆永洋一},山本芳嗣 (2014):\\
{\bf モジュラリティ最大化問題に対する切除平面法に基づく発見的解法.}\\
{数理解析研究所講究録}, 1879巻, pp.15-27.

\item 正木俊行,\underline{伊豆永洋一},佐藤俊樹,鮭川矩義,石濱友裕,田中彰浩,中島雄基,舟橋史明 (2013):\\
{\bf アクセスログデータ可視化の試み.}\\
{オペレーションズ・リサーチ経営の科学}, Vol.58, No.2, pp.74-79.

\end{enumerate}

%---talks(international)---
\section*{\underline{国際会議発表}}

\begin{enumerate}

\item Yoichi Izunaga:\\
{\bf A Lagrangian relaxation algorithm for modularity maximization.}\\
{27th European Conference on Operational Research} (2015).

\item Yoichi Izunaga:\\
{\bf Row and column generation algorithm for maximization of minimum margin for ranking problem.}\\
{International Conference on Operations Research 2014} (2014).


\item Yoichi Izunaga:\\
{\bf A cutting plane algorithm for modularity maximization with heuristics for separation problem.}\\
Tsukuba Global Science Week~2013 (2013).

\end{enumerate}

%---talks(domestic)---
\section*{\underline{国内学会等発表}}

\begin{enumerate}

\item 伊豆永洋一:\\
{\bf モジュラリティデンシティ最大化問題に対する再定式化.}\\
{科学研究費 基盤研究(A)「新時代の最適化モデルに基づく意思決定支援プラットフォームの研究と開発」2018年度ワークショップ}, (2018, 招待講演).

\item 伊豆永洋一:\\
{\bf モジュラリティデンシティ最大化問題に対する再定式化とアルゴリズム.}\\
{日本オペレーションズ・リサーチ学会 「最適化の基盤とフロンティア」研究部会 第14回研究会}, 東京理科大学 (2017, 招待講演).

\item 伊豆永洋一:\\
{\bf A bounding algorithm for selective graph coloring problem.}\\
{京都大学数理解析研究所研究集会「数理最適化の発展: モデル化とアルゴリズム」}, 京都大学 (2017).

\item 伊豆永洋一:\\
{\bf A doubly nonnegative relaxation for modularity density maximization.}\\
{スペクトラルグラフ理論および周辺領域 第4回研究集会}, 筑波大学 (2015).

\item 伊豆永洋一:\\
{\bf モジュラリティデンシティ最大化問題に対する非負半正定値緩和.}\\
{京都大学数理解析研究所研究集会「新時代を担う最適化: モデル化手法と数値計算」}, 京都大学 (2015).

\item 伊豆永洋一:\\
{\bf モジュラリティデンシティ最大化問題に対する錐計画緩和.}\\
{組合せ数学セミナー}, 東京大学 (2015, 招待講演).

\item 伊豆永洋一:\\
{\bf モジュラリティデンシティ最大化問題に対する非負半正定値緩和.}\\
{日本オペレーションズ・リサーチ学会 「最適化の基盤とフロンティア」研究部会-未来を担う研究者の集い2015-}, 筑波大学 (2015).

\item 伊豆永洋一:\\
{\bf Row and column generation algorithm for minimum margin maximization of ranking problems.}\\
{日本オペレーションズ・リサーチ学会常設研究部会「評価のOR」第63回研究会}, 東京理科大学 (2015, 招待講演).

\item 伊豆永洋一:\\
{\bf Row and column generation algorithm for minimum margin maximization of ranking problems.}\\
{京都大学数理解析研究所研究集会「最適化アルゴリズムの進展: 理論・応用・実装」}, 京都大学 (2014).

\item 伊豆永洋一:\\
{\bf Row and column generation algorithm for minimum margin maximization for ranking problem.}\\
{日本オペレーションズ・リサーチ学会 「最適化の理論と応用」研究部会-未来を担う研究者の集い2014-}, 筑波大学 (2014).

\item 伊豆永洋一:\\
{\bf モジュラリティ最大化問題に対する切除平面法に基づく発見的解法.}\\
{日本オペレーションズ・リサーチ学会秋季研究発表会}, 徳島大学 (2013).

\item 伊豆永洋一:\\
{\bf モジュラリティ最大化問題に対する切除平面法に基づく発見的解法.}\\
{京都大学数理解析研究所研究集会「最適化の基礎理論と応用」}, 京都大学 (2013).

\item 伊豆永洋一:\\
{\bf モジュラリティ最大化問題に対する列生成法に基づく発見的解法.}\\
{日本オペレーションズ・リサーチ学会 「最適化の理論と応用」研究部会-未来を担う研究者の集い2013-}, 筑波大学 (2013).

\item 伊豆永洋一:\\
{\bf An experimental evaluation of an approximation algorithm for single-item multi-unit auctions.}\\
{第16回実験社会科学カンファレンス}, 青山学院大学 (2012).

\item 伊豆永洋一:\\
{\bf An experimental evaluation of an approximation algorithm for single-item multi-unit auctions.}\\
{The 18th Decentralization Conference}, 関西大学 (2012).

\end{enumerate}


%---grant---
\section*{\underline{外部資金獲得実績}}

\begin{enumerate}
\item 2015年: 公益財団法人 村田学術振興財団 平成27年度研究者海外派遣援助\\
{モジュラリティ最大化問題に対するラグランジュ緩和に基づくアルゴリズム.}
\end{enumerate}




\end{document}